\documentclass[a4paper, 12pt]{article}				

%============== Русский язык ===============================
\usepackage[T2A]{fontenc}		
\usepackage[utf8]{inputenc}	
\usepackage[english,russian]{babel}	
\usepackage{listings,  float, amsmath, amsthm, amssymb, cmap, graphicx, xcolor, hyperref, geometry}
\usepackage{amsfonts}
\usepackage{longtable}
\usepackage{epsfig}
\usepackage{verbatim}
\theoremstyle{plain}

\usepackage{graphicx}
\usepackage{caption}
\usepackage{subcaption}
\usepackage{float}
\restylefloat{table}
\usepackage{tcolorbox}

\theoremstyle{problem}
\newtheorem{problem}{Задача}




\title{Поиск вектора, максимизирующего скалярное произведение с заданным запросом}

\date{}


\begin{document}
\maketitle
\begin{problem}
По множеству точек $X \subset \mathbb{R}^d$ и запросу $q \in \mathbb{R}^d$ найти точку $p \in X$, т.ч:
$$p = \textit{argmax}_{X}{<q, x>}$$
\end{problem}


\subsubsection*{Приближенная переформулировка для случая известного множества части запросов}
$X = \{x_1, ... x_n\}$ --- выборка. $x_i \in \mathbb{R}^d$\\
$Q = \{q_1, ... q_m\}$ --- примеры запросов. $q_i \in \mathbb{R}^d$\\
Требуется найти разбиение $X = X_1 \sqcup X_2$, которое бы в среднем неплохо отделяло вектора с маленьким скалярным произведением от векторов с большим.
$$\sum_{q \in Q}\left(\frac{1}{||X_1||}\sum_{x_1 \in X_1}<q,x_1> - \frac{1}{||X_2||}\sum_{x_2 \in X_2}<q, x_2>\right)^2 \longrightarrow \max_{X_1, X_2}$$
Добавляя условие сбалансированности разделения ($||X_1|| = ||X_2||$), можем перейти к следующей задаче

$$\sum_{Q}\left(\sum_{X_1}<q,x_1> - \sum_{X_2}<q, x_2>\right)^2 \longrightarrow \max_{X_1, X_2, ||X_1|| = ||X_2||}$$
Упростим немного выражение слева
\begin{align*}
&\sum_{Q}\left(\sum_{X_1}<q,x_1> - \sum_{X_2}<q, x_2>\right)^2  =\\= 
&\sum_Q\left(<q, \sum_{X_1}x_1 - \sum_{X_2}x_2>\right)^2 =\\=
&\sum_Q (\sum_{X_1}x_1 - \sum_{X_2}x_2)^Tqq^T(\sum_{X_1}x_1 - \sum_{X_2}x_2) =\\=
&(\sum_{X_1}x_1 - \sum_{X_2}x_2)^TQ^TQ(\sum_{X_1}x_1 - \sum_{X_2}x_2) = \\=
&(\sum_{X_1}x_1 - \sum_{X_2}x_2)^TW^TW(\sum_{X_1}x_1 - \sum_{X_2}x_2) = \\=
&(\sum_{X_1}Wx_1 - \sum_{X_2}Wx_2)^T(\sum_{X_1}Wx_1 - \sum_{X_2}Lx_2) = \\=
&||\sum_{X_1}Wx_1 - \sum_{X_2}Wx_2||^2 
\end{align*}
$W \in \mathbb{R}^{r \times d}$, где $r \leq d$.\\
Если $m \leq d$, то, в качестве $W$ можно взять саму $Q$. \\
Если $m \gg d$, то сойдет разложение Холецкого.
\begin{tcolorbox}
$$||\sum_{X_1}Wx_1 - \sum_{X_2}Wx_2||^2\longrightarrow \max_{\{X_1, X_2 | X_1 \sqcup X_2 = X, ||X_1|| = ||X_2||\}}$$
\end{tcolorbox}
\end{document}